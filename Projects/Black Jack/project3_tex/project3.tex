%----------------------------------------------------------------------------------------
%	PACKAGES AND OTHER DOCUMENT CONFIGURATIONS
%----------------------------------------------------------------------------------------

\documentclass[paper=a4, fontsize=10pt]{scrartcl} % A4 paper and 10pt font size

\usepackage[T1]{fontenc} % Use 8-bit encoding that has 256 glyphs
%\usepackage{fourier} % Use the Adobe Utopia font for the document - comment this line to return to the LaTeX default
\usepackage[english]{babel} % English language/hyphenation
\usepackage{amsmath,amsfonts,amsthm} % Math packages
\usepackage{algorithm}
%\usepackage{algorithmic}
\usepackage{algpseudocode}
\renewcommand{\algorithmicrequire}{\textbf{Input:}} % Use Input in the format of Algorithm
\renewcommand{\algorithmicensure}{\textbf{Output:}} % Use Output in the format of Algorithm
\usepackage{color}
\usepackage[colorlinks, linkcolor=Navy]{hyperref}


\usepackage{geometry}
\geometry{scale=0.8}

 \usepackage{setspace}



\usepackage{lipsum} % Used for inserting dummy 'Lorem ipsum' text into the template

\usepackage{sectsty} % Allows customizing section commands
%\allsectionsfont{\centering \normalfont\scshape} % Make all sections centered, the default font and small caps

\usepackage{fancyhdr} % Custom headers and footers
\pagestyle{fancyplain} % Makes all pages in the document conform to the custom headers and footers
\fancyhead{} % No page header - if you want one, create it in the same way as the footers below
\fancyfoot[L]{} % Empty left footer
\fancyfoot[C]{} % Empty center footer
\fancyfoot[R]{\thepage} % Page numbering for right footer
\renewcommand{\headrulewidth}{0pt} % Remove header underlines
\renewcommand{\footrulewidth}{0pt} % Remove footer underlines
\setlength{\headheight}{0pt} % Customize the height of the header

\numberwithin{equation}{section} % Number equations within sections (i.e. 1.1, 1.2, 2.1, 2.2 instead of 1, 2, 3, 4)
\numberwithin{figure}{section} % Number figures within sections (i.e. 1.1, 1.2, 2.1, 2.2 instead of 1, 2, 3, 4)
\numberwithin{table}{section} % Number tables within sections (i.e. 1.1, 1.2, 2.1, 2.2 instead of 1, 2, 3, 4)

%\setlength\parindent{0pt} % Removes all indentation from paragraphs - comment this line for an assignment with lots of text

%----------------------------------------------------------------------------------------
%	TITLE SECTION
%----------------------------------------------------------------------------------------

\newcommand{\horrule}[1]{\rule{\linewidth}{#1}} % Create horizontal rule command with 1 argument of height

\title{	
\normalfont \normalsize 
\textsc{Fudan University, Artificial Intelligence} \\ [0pt] % Your university, school and/or department name(s)
\horrule{1pt} \\[0.4cm] % Thin top horizontal rule
\LARGE Report on Project 3: Black Jack \\ % The assignment title
\horrule{1pt} \\[0cm] % Thick bottom horizontal rule
}

\author{Ao Wang, 15300240004} % Your name

\date{\normalsize\today} % Today's date or a custom date

\begin{document}

\begin{spacing}{1.}

\maketitle % Print the title

%----------------------------------------------------------------------------------------
\section{Overview}
This project is about solving Black Jack using MDPs. This pdf contains the answers of the required questions.

\section{Problem 1}
\subsection{Problem 1a}
As required by the instructions, the state -2 and 2 are the exits, so we manually set the $V_{opt}(s)$ of them to a fixed number, which is 0. Then the results are as follows:
\begin{table}[h]
\setlength{\abovecaptionskip}{0.cm}
\setlength{\belowcaptionskip}{-0.cm}
\centering
  \caption{The result of 1st value iteration (after 0 iterations).}
  %\label{tab:test}
  \begin{tabular}{|c|c|c|c|c|c|}
     \hline
    State $s$ & -2 & -1 & 0 & 1 & 2\\
     \hline
    $V_{opt}(s)$ & 0.0 & \textbf{11.0} & \textbf{-5.0} & \textbf{25.0} & 0.0\\
     \hline
  \end{tabular}
\end{table}

\begin{table}[h]
\setlength{\abovecaptionskip}{0.cm}
\setlength{\belowcaptionskip}{-0.cm}
\centering
  \caption{The result of 2nd value iteration (after 1 iterations).}
  %\label{tab:test}
  \begin{tabular}{|c|c|c|c|c|c|}
     \hline
    State $s$ & -2 & -1 & 0 & 1 & 2\\
     \hline
    $V_{opt}(s)$ & 0.0 & \textbf{10.0} & \textbf{10.2} & \textbf{21.5} & 0.0\\
     \hline
  \end{tabular}
\end{table}

\begin{table}[h]
\setlength{\abovecaptionskip}{0.cm}
\setlength{\belowcaptionskip}{-0.cm}
\centering
  \caption{The result of 3rd value iteration (after 2 iterations).}
  %\label{tab:test}
  \begin{tabular}{|c|c|c|c|c|c|}
     \hline
    State $s$ & -2 & -1 & 0 & 1 & 2\\
     \hline
    $V_{opt}(s)$ & 0.0 & \textbf{13.04} & \textbf{8.45} & \textbf{32.14} & 0.0\\
     \hline
  \end{tabular}
\end{table}

\subsection{Problem 1b}
After the iteration is converged, the result of $\pi_{opt}$ is as follows:
\begin{table}[!htbp]
\setlength{\abovecaptionskip}{0.cm}
\setlength{\belowcaptionskip}{-0.cm}
\centering
  \caption{The result of the optimal actions after being converged.}
  %\label{tab:test}
  \begin{tabular}{|c|c|c|c|c|c|}
     \hline
    State $s$ & -2 & -1 & 0 & 1 & 2\\
     \hline
    $\pi_{opt}(s)$ & - & \textbf{-1} & \textbf{+1} & \textbf{+1} & -\\
     \hline
  \end{tabular}
\end{table}

The result is pretty clear because the agent tries to get to the exit as soon as possible as the reward of each step is negative and both of the exits have comparatively high positive rewards. But at state 0, agent will have higher expectation by moving +1, although it has a high possibility of moving -1.

\section{Problem 2}
\subsection{Problem 2a}
According to the instruction, the transition (probability) has been changed with noise, with a probability to randomly change to a possible reachable state. We can find a counter example that $V_{1}(s_{start}) \le V_{2}(s_{start})$. 

%----------------------------------------------------------------------------------------

\end{spacing}
\end{document}