%----------------------------------------------------------------------------------------
%	PACKAGES AND OTHER DOCUMENT CONFIGURATIONS
%----------------------------------------------------------------------------------------

\documentclass[paper=a4, fontsize=10pt]{scrartcl} % A4 paper and 10pt font size

\usepackage[T1]{fontenc} % Use 8-bit encoding that has 256 glyphs
%\usepackage{fourier} % Use the Adobe Utopia font for the document - comment this line to return to the LaTeX default
\usepackage[english]{babel} % English language/hyphenation
\usepackage{amsmath,amsfonts,amsthm} % Math packages
\usepackage{algorithm}
%\usepackage{algorithmic}
\usepackage{algpseudocode}
\renewcommand{\algorithmicrequire}{\textbf{Input:}} % Use Input in the format of Algorithm
\renewcommand{\algorithmicensure}{\textbf{Output:}} % Use Output in the format of Algorithm
\usepackage{color}
\usepackage[colorlinks, linkcolor=black]{hyperref}
\usepackage{graphicx}
\usepackage{subfigure}
\usepackage{color}
\usepackage{amsfonts,amssymb}


\usepackage{geometry}
\geometry{scale=0.8}

 \usepackage{setspace}



\usepackage{lipsum} % Used for inserting dummy 'Lorem ipsum' text into the template

\usepackage{sectsty} % Allows customizing section commands
%\allsectionsfont{\centering \normalfont\scshape} % Make all sections centered, the default font and small caps

\usepackage{fancyhdr} % Custom headers and footers
\pagestyle{fancyplain} % Makes all pages in the document conform to the custom headers and footers
\fancyhead{} % No page header - if you want one, create it in the same way as the footers below
\fancyfoot[L]{} % Empty left footer
\fancyfoot[C]{} % Empty center footer
\fancyfoot[R]{\thepage} % Page numbering for right footer
\renewcommand{\headrulewidth}{0pt} % Remove header underlines
\renewcommand{\footrulewidth}{0pt} % Remove footer underlines
\setlength{\headheight}{0pt} % Customize the height of the header

\numberwithin{equation}{section} % Number equations within sections (i.e. 1.1, 1.2, 2.1, 2.2 instead of 1, 2, 3, 4)
\numberwithin{figure}{section} % Number figures within sections (i.e. 1.1, 1.2, 2.1, 2.2 instead of 1, 2, 3, 4)
\numberwithin{table}{section} % Number tables within sections (i.e. 1.1, 1.2, 2.1, 2.2 instead of 1, 2, 3, 4)

%\setlength\parindent{0pt} % Removes all indentation from paragraphs - comment this line for an assignment with lots of text

%----------------------------------------------------------------------------------------
%	TITLE SECTION
%----------------------------------------------------------------------------------------

\newcommand{\horrule}[1]{\rule{\linewidth}{#1}} % Create horizontal rule command with 1 argument of height

\title{	
\normalfont \normalsize 
\textsc{Fudan University, Artificial Intelligence} \\ [0pt] % Your university, school and/or department name(s)
\horrule{1pt} \\[0.4cm] % Thin top horizontal rule
\LARGE Report on Project 4: Driverless Car \\ % The assignment title
\horrule{1pt} \\[0cm] % Thick bottom horizontal rule
}

\author{Ao Wang, 15300240004} % Your name

\date{\normalsize\today} % Today's date or a custom date

\begin{document}

\begin{spacing}{1.}

\maketitle % Print the title

%----------------------------------------------------------------------------------------

\section{Overview}
In this project, we use HMM to solve driverless car problems.

\section{Problem 1}
\subsection{Problem 1a}
According to the constraints given by the instruction,
\begin{displaymath}
\begin{aligned}
\mathbb{P}(C_{2}=1|D_{2}=0) & = \frac{\mathbb{P}(C_{2}=1, D_{2}=0)}{\mathbb{P}(D_{2}=0)} \\
&= \frac{\mathbb{P}(D_{2}=0|C_{2}=1) \mathbb{P}(C_{2}=1)}{\mathbb{P}(D_{2}=0)}\\
\end{aligned}
\end{displaymath}

And we have
\begin{displaymath}
\begin{aligned}
\mathbb{P}(C_{2}=1) &= \mathbb{P}(C_{1}=0)\mathbb{P}(C_{2}=1|C_{1}=0) + \mathbb{P}(C_{1}=1)\mathbb{P}(C_{2}=1|C_{1}=1)\\
& = 0.5 \epsilon + 0.5 (1 - \epsilon)\\
& = 0.5
\end{aligned}
\end{displaymath}

So we have
\begin{displaymath}
\begin{aligned}
\mathbb{P}(C_{2}=1|D_{2}=0) &= \frac{0.5 \eta}{\mathbb{P}(D_{2}=0)}\\
\mathbb{P}(C_{2}=0|D_{2}=0) &= \frac{0.5 (1 - \eta)}{\mathbb{P}(D_{2}=0)}
\end{aligned}
\end{displaymath}

So the answer is
\begin{displaymath}
\begin{aligned}
\mathbb{P}(C_{2}=1|D_{2}=0) &= \frac{0.5 \eta}{0.5 \eta + 0.5 (1 - \eta)}\\
&= \eta
\end{aligned}
\end{displaymath}

Notice that $\mathbb{P}(D_{2}=0)$ is not 1 since $\mathbb{P}(D_{2})$ is related to $\mathbb{P}(C_{2})$.

\subsection{Problem 1b}
According to the HMM graph, the result could be written as
\begin{displaymath}
\begin{aligned}
&\ \ \ \ \mathbb{P}(C_{2}=1|D_{2}=0, D_{3}=1) \\
&\propto  \mathbb{P}(C_{2}=1, D_{2}=0, D_{3}=1)\\
&=\{\sum_{C_{1} = \{0, 1\}}\mathbb{P}(C_{1})\mathbb{P}(C_{2} = 1|C_{1})\} \cdot \mathbb{P}(D_{2}=0|C_{2}=1) \cdot \{\sum_{C_{3} = \{0, 1\}}\mathbb{P}(C_{3}|C_{2}=1)\mathbb{P}(D_{3}=1|C_{3})\}\\
& = 0.5\eta(\epsilon\eta+(1-\epsilon)(1-\eta))
\end{aligned}
\end{displaymath}

Also,
\begin{displaymath}
\begin{aligned}
&\ \ \ \ \mathbb{P}(C_{2}=0|D_{2}=0, D_{3}=1) \\
&\propto  \mathbb{P}(C_{2}=0, D_{2}=0, D_{3}=1)\\
&=\{\sum_{C_{1} = \{0, 1\}}\mathbb{P}(C_{1})\mathbb{P}(C_{2} = 0|C_{1})\} \cdot \mathbb{P}(D_{2}=0|C_{2}=0) \cdot \{\sum_{C_{3} = \{0, 1\}}\mathbb{P}(C_{3}|C_{2}=0)\mathbb{P}(D_{3}=1|C_{3})\}\\
& = 0.5(1-\eta)((1-\epsilon)\eta+\epsilon(1-\eta))
\end{aligned}
\end{displaymath}

After normalization, the result is 
\begin{displaymath}
\begin{aligned}
\mathbb{P}(C_{2}=1|D_{2}=0, D_{3}=1) = \frac{\epsilon\eta^{2} + \eta(1-\epsilon)(1-\eta)}{\epsilon\eta^{2} + 2\eta(1-\epsilon)(1-\eta) + \epsilon(1-\eta)^{2}}
\end{aligned}
\end{displaymath}

\subsection{Problem 1c}
\subsubsection{i}









%----------------------------------------------------------------------------------------

\end{spacing}
\end{document}